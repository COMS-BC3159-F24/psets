\documentclass[]{article}

\usepackage{tcolorbox, booktabs, tikz, tikzsymbols, 
            lmodern, amssymb, amsmath, ifxetex, ifluatex, 
            hyperref, listings, float, graphicx, caption,
            fancyhdr, framed, xcolor}

\usepackage[left=3cm,right=3cm,top=2cm,bottom=2cm]{geometry}

% Box Macros
\DeclareRobustCommand{\colorbox}[3][gray!10]{
  \begin{tcolorbox}[left=8pt, arc=0pt, outer arc=0pt, colframe=#1, colback=#1, coltext=black] #2 \end{tcolorbox}
}

\DeclareRobustCommand{\solutionbox}[2][gray!20]{%
\begin{tcolorbox}[left=2pt, arc=0pt, outer arc=0pt, colframe = gray!10] \begin{center} #2 \end{center} \end{tcolorbox}}

% Paragraphs
\setlength{\parindent}{0pt}
\setlength{\parskip}{6pt plus 2pt minus 1pt}

% No monospace font for urls
\urlstyle{same}  
\usepackage{hyperref}
\hypersetup{
            colorlinks=true,
            linkcolor=Maroon,
            filecolor=Maroon,
            citecolor=Blue,
            urlcolor=blue,
            breaklinks=true
}

% Code formatting
\usepackage[dvipsnames]{xcolor}
\lstset{
  language=python,
  basicstyle=\ttfamily,
  commentstyle=\color{BrickRed},
  stringstyle=\color{ForestGreen},
  showstringspaces=false,
  keywordstyle=\color{Purple},
  emph={def},
  emphstyle=\color{RoyalBlue}
}


\fancypagestyle{default}{
    \lhead{Problem Set 1}
    \chead{}
    \rhead{\textbf{COMS BC3159}}
    \lfoot{}
    \cfoot{}
    \rfoot{\thepage}
    \renewcommand{\headrulewidth}{0.4pt}
    \renewcommand{\footrulewidth}{0.4pt}
}

\pagestyle{default}% Default page style
\fancypagestyle{plain}{\pagestyle{default}}


\title{\textbf{COMS BC3159:} Problem Set 1}
\author{\textbf{Due} Friday, September 27, 2024 11:59 PM\\
    %%%%%%%%%%%%%%%%%%%%%%%%%%%%%%%%%%%%%%%%%
    %                                       %
    % TODO: Your Name Here                  %
    %                                       %
    %%%%%%%%%%%%%%%%%%%%%%%%%%%%%%%%%%%%%%%%%
}
\date{}

\begin{document}
\maketitle

\colorbox {
Please show your work for all solutions to receive partial/full credit.  Always turn in \textit{only} your own, independent work to \href{https://www.gradescope.com/courses/820552}{\textbf{Gradescope}} (assign each question to a page in your submission).  For our late policy, refer to the syllabus.  All other questions can be posted on Slack \#ps1.
}

\vspace{0.5cm}

\textbf{Collaborators:}
%%%%%%%%%%%%%%%%%%%%%%%%%%%%%%%%%%%%%%%%%
%                                       %
% TODO: Names of any Collaborators Here %
%                                       %
%%%%%%%%%%%%%%%%%%%%%%%%%%%%%%%%%%%%%%%%%
\\ \\
\textbf{AI Tool Disclosure:}
%%%%%%%%%%%%%%%%%%%%%%%%%%%%%%%%%%%%%%%%%
%                                       %
% TODO: How did you use AI tools?       %
%                                       %
%%%%%%%%%%%%%%%%%%%%%%%%%%%%%%%%%%%%%%%%%

\vspace{0.5cm}
\hrule
\vspace{0.5cm}
The following exercises will explore computer systems and architecture as well as GPU parallel programming. In addition to these written problems, you will also complete the coding part of the assignment, found in the GitHub repo. 
\vspace{0.5cm}
\hrule

\section{Let's talk cache (6 Points)}
Consider the following function which sums every other value in an array:
\begin{tcolorbox}[left=14pt, arc=0pt, outer arc=0pt, colframe=blue!5, colback=blue!5]
\begin{lstlisting}[language=C]
void sum_even_vector(int *array) {
    int sum = 0;
    for (int i = 0; i < N; i += 2) {
        sum += array[i];
    }
    return sum;
}
\end{lstlisting}
\end{tcolorbox}
Take these assumptions:
\begin{itemize}
    \item When you access memory from RAM, you load 64 bytes of data into the cache.
    \item Loading from RAM to cache and then into registers takes a total of 26ns.
    \item Future loads of this data from the cache into registers takes only 2ns.
    \item All computations and loads of data back into cache and registers are free (aka it takes 0ns). Aka we are only worried (for this question) about the time taken to read the raw data from disk!
\end{itemize}

\newpage

For each scenario, how long would it take to load data into registers during the runtime of the function? Please write your answer as a function of \texttt{N} (where \texttt{N} is the total number of elements in the array).

\begin{enumerate}
    \item[(a)] First, consider the case where this program is run in one thread.
    \item[(b)] Second, consider the case where the for loop in lines 3-5 is parallelized across 4 blocks on the GPU each with one thread evenly distributing the work. Ignore thread launch or synchronizations overheads. Can you describe the least efficient memory access pattern for the four threads? How long would it take? Is this faster or slower than the serial case?
    \item[(c)] Conversely in the multi-threaded case, can you describe the most efficient memory access pattern? How long would it take? Is this faster or slower than the serial case?
\end{enumerate}
\textit{Hint: integers are 4 bytes in size!}

\bigskip
%%%%%%%%%%%%%%%%%%%%%%%%%%%%%%%%%%%%%%%%%
%                                       %
% TODO: Your solution to Problem 1      %
%                                       %
%%%%%%%%%%%%%%%%%%%%%%%%%%%%%%%%%%%%%%%%%

% \begin{enumerate}
%     \item[(a)]
%     \item[(b)]
%     \item[(c)]
% \end{enumerate}

%%%%%%%%%%%%%%%%%%%%%%%%%%%%%%%%%%%%%%%%%
%%%%%%%%%%%%%%%%%%%%%%%%%%%%%%%%%%%%%%%%%
%%%%%%%%%%%%%%%%%%%%%%%%%%%%%%%%%%%%%%%%%

\newpage

\section{SIMD, MIMD, \& Everything in Between (4 points)}
Are the following statements true or false? Please explain why in 1-3 sentences.

\begin{enumerate}
    \item[(a)] A sorting algorithm which divides an array into four parts and has each part sorted on a different physical computer using a different sorting algorithm (quick sort, merge sort, bubble sort, etc.) is an example of a MIMD computational pattern.
    \item[(b)] An algorithm that compute the sum of two arrays using a serial loop on a single thread is an example of a SIMD computational pattern.
    \item[(c)] All threads launched in a computational grid (aka all of the threads across all of the blocks launched as a part of a single GPU kernel) can access the same shared memory.
    \item[(d)] You cannot control which SM a thread block runs on in a GPU.
\end{enumerate}

\bigskip
%%%%%%%%%%%%%%%%%%%%%%%%%%%%%%%%%%%%%%%%%
%                                       %
% TODO: Your solution to Problem 2      %
%                                       %
%%%%%%%%%%%%%%%%%%%%%%%%%%%%%%%%%%%%%%%%%

% \begin{enumerate}
%     \item[(a)] \textbf{True/False}, explanation...
%     \item[(b)] \textbf{True/False}
%     \item[(c)] \textbf{True/False}
%     \item[(d)] \textbf{True/False}
% \end{enumerate}

%%%%%%%%%%%%%%%%%%%%%%%%%%%%%%%%%%%%%%%%%
%%%%%%%%%%%%%%%%%%%%%%%%%%%%%%%%%%%%%%%%%
%%%%%%%%%%%%%%%%%%%%%%%%%%%%%%%%%%%%%%%%%


\newpage

\section{CUDA Programming}
\subsection{Some mistakes (6 Points + 1 Bonus)}
Consider this CUDA program intended to sum the numbers from 1 to $N$. Unfortunately, there are a few mistakes preventing the code from running correctly and producing correct output. List all the changes that need to be made to fix this code.

\colorbox[SeaGreen!10]{You should find at least 3 major issues for full credit. One bonus point if you find all four of the major issues (the last one is a bit more subtle).}

\begin{tcolorbox}[left=14pt, arc=0pt, outer arc=0pt, colframe=blue!5, colback=blue!5]
\begin{lstlisting}[language=C]
#include <iostream>
#include <cuda_runtime.h>

const int N = 100;

// add 1 to all values in an array
__global__
void increment_kernel(int *d_data, int N) {
    int idx = threadIdx.x;
    if (idx < N) {d_data[idx] += 1;}
}

__host__
int main() {
    int *h_data = (int *)malloc(N*sizeof(int));
    int *d_data; cudaMalloc(&d_data, N * sizeof(int));

    // initialize the data
    for (int i =0; i < N; i++){h_data[i] = i;}

    // apply the kernel
    increment_kernel<<<10, 10>>>(d_data, N);
    cudaDeviceSynchronize();

    // sum the resulting array
    int sum = 0;
    for (int i = 0; i < N; i++) {
        sum += d_data[i];
    }
    std::cout << "Sum: " << sum << std::endl;

    free(h_data); cudaFree(d_data);
    return 0;
}
\end{lstlisting}
\end{tcolorbox}

%%%%%%%%%%%%%%%%%%%%%%%%%%%%%%%%%%%%%%%%%
%                                       %
% TODO: Your solution to Problem 3.1    %
%                                       %
%%%%%%%%%%%%%%%%%%%%%%%%%%%%%%%%%%%%%%%%%

% Things to fix ...
% \begin{enumerate}
%     \item
%     \item
%     \item
% \end{enumerate}

%%%%%%%%%%%%%%%%%%%%%%%%%%%%%%%%%%%%%%%%%
%%%%%%%%%%%%%%%%%%%%%%%%%%%%%%%%%%%%%%%%%
%%%%%%%%%%%%%%%%%%%%%%%%%%%%%%%%%%%%%%%%%

\newpage
\subsection{CUDA, again (4 Points)}
Consider \textit{another} CUDA program intended to efficiently compute the sum of an array.
\begin{itemize}
    \item [(a)] If you made no changes to this code, not only would it not produce the correct answer, but it would never finish running. Why?
    \item [(b)] This code has a number of missing or misplaced calls to synchronization primitives. Please describe where and how they should be added / changed / moved.
    \begin{tcolorbox}[left=14pt, arc=0pt, outer arc=0pt, colframe=SeaGreen!10, colback=SeaGreen!10]
    \textit{Hint:} You should be able to find at least 2 places to add / change / move them.
    \end{tcolorbox}
\end{itemize}


\begin{tcolorbox}[left=14pt, arc=0pt, outer arc=0pt, colframe=blue!5, colback=blue!5]
\begin{lstlisting}[language=C]
    #include <stdio.h>
    #include <cuda_runtime.h>
    
    __global__ 
    void reduce(int *d_input, int *d_output, int n) {
        // define our shared (cache) memory
        extern __shared__ int s_data[];
        
        // Load data into shared memory from RAM
        for (int i = threadIdx.y; i < n; i++) {
            s_data[i] = d_input[i];
        }
    
        // Reduction in shared memory
        while(n > 1) {
            for(int i=threadIdx.x; i < n/2; i += blockDim.x) {
                s_data[i] += s_data[i + n/2];
                __syncthreads();
            }
            // account for odd length arrays
            if (n % 2 && threadIdx.x == 0) {
                s_data[0] += s_data[n-1];
            } 
            // update tree layer
            n = n/2;
        }
        // save back to RAM from shared
        if (threadIdx.x == 0) {
            d_output[0] = s_data[0];
        }
    }
\end{lstlisting}
\end{tcolorbox}

\begin{tcolorbox}[left=14pt, arc=0pt, outer arc=0pt, colframe=blue!5, colback=blue!5]
\begin{lstlisting}[language=C]
    __host__
    int main() {
        const int n = 64;
    
        // initialize the values to 1
        int h_vals[n];
        for(int i = 0; i < n; i++) {
            h_vals[i] = 1;
        }
        
        // copy onto the GPU
        int nInt = n * sizeof(int);
        int *d_vals; cudaMalloc((void**) &d_vals, nInt);
        cudaMemcpy(d_vals, h_vals, 
                   nInt, cudaMemcpyHostToDevice);
        
        // allocate memory for the sum
        int h_sum; int *d_sum; 
        cudaMalloc((void**) &d_sum, sizeof(int));
    
        // run the kernel
        int nThreads = 256;
        reduce<<<1, nThreads, nThreads * sizeof(int)>>>(d_vals,
                                                        d_sum, 
                                                        n);
        cudaDeviceSynchronize();
        
        // copy the result back and display
        cudaMemcpy(&h_sum, d_sum,
                   sizeof(int), cudaMemcpyDeviceToHost);
        printf("%d\n", h_sum);
    
        // free vars and exit
        cudaFree(d_vals); 
        cudaFree(d_sum); 
        return 0;
    }
\end{lstlisting}
\end{tcolorbox}

%%%%%%%%%%%%%%%%%%%%%%%%%%%%%%%%%%%%%%%%%
%                                       %
% TODO: Your solution to Problem 3.2    %
%                                       %
%%%%%%%%%%%%%%%%%%%%%%%%%%%%%%%%%%%%%%%%%

% \textbf{Part A:}

% \textbf{Part B:}

%%%%%%%%%%%%%%%%%%%%%%%%%%%%%%%%%%%%%%%%%
%%%%%%%%%%%%%%%%%%%%%%%%%%%%%%%%%%%%%%%%%
%%%%%%%%%%%%%%%%%%%%%%%%%%%%%%%%%%%%%%%%%

\vfill
\begin{tiny}Updated \today \end{tiny}
\end{document}