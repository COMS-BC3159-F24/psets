\documentclass[]{article}

\usepackage{tcolorbox, booktabs, tikz, tikzsymbols, 
            lmodern, amssymb, amsmath, ifxetex, ifluatex, 
            hyperref, listings, float, graphicx, caption,
            fancyhdr, framed, xcolor}

\usepackage[left=3cm,right=3cm,top=2cm,bottom=2cm]{geometry}

% Box Macros
\DeclareRobustCommand{\colorbox}[3][gray!10]{
  \begin{tcolorbox}[left=8pt, arc=0pt, outer arc=0pt, colframe=#1, colback=#1, coltext=black] #2 \end{tcolorbox}
}

\DeclareRobustCommand{\solutionbox}[2][gray!20]{%
\begin{tcolorbox}[left=2pt, arc=0pt, outer arc=0pt, colframe = gray!10] \begin{center} #2 \end{center} \end{tcolorbox}}

% Paragraphs
\setlength{\parindent}{0pt}
\setlength{\parskip}{6pt plus 2pt minus 1pt}

% No monospace font for urls
\urlstyle{same}  
\usepackage{hyperref}
\hypersetup{
            colorlinks=true,
            linkcolor=Maroon,
            filecolor=Maroon,
            citecolor=Blue,
            urlcolor=blue,
            breaklinks=true
}

% Code formatting
\usepackage[dvipsnames]{xcolor}
\lstset{
  language=python,
  basicstyle=\ttfamily,
  commentstyle=\color{BrickRed},
  stringstyle=\color{ForestGreen},
  showstringspaces=false,
  keywordstyle=\color{Purple},
  emph={def},
  emphstyle=\color{RoyalBlue}
}


\fancypagestyle{default}{
    \lhead{Problem Set 0}
    \chead{}
    \rhead{\textbf{COMS BC3159}}
    \lfoot{}
    \cfoot{}
    \rfoot{\thepage}
    \renewcommand{\headrulewidth}{0.4pt}
    \renewcommand{\footrulewidth}{0.4pt}
}

\pagestyle{default}% Default page style
\fancypagestyle{plain}{\pagestyle{default}}


\title{\textbf{COMS BC3159:} Problem Set 0}
\author{\textbf{Due} Friday, September 13, 2024 11:59 PM\\
    %%%%%%%%%%%%%%%%%%%%%%%%%%%%%%%%%%%%%%%%%
    %                                       %
    % TODO: Your Name Here                  %
    %                                       %
    %%%%%%%%%%%%%%%%%%%%%%%%%%%%%%%%%%%%%%%%%
}
\date{}

\begin{document}
\maketitle

\colorbox[blue!5] {
Hello and welcome to COMS BC3159!
\begin{center}{\fontfamily{bch}\selectfont ``Ugh, why do I have to do PS0?  We haven't even learned anything yet!?''}
\end{center}
Take this opportunity to \textbf{practice} completing and turning in assignments for our class.  After PS0, we will expect you to know the infrastructure for submissions.\\

PS0 should \textbf{not} feel arduous nor time consuming---if you are struggling with this assignment, please do reach out to teaching staff or other classmates!
}

\colorbox {
Please show your work for all solutions to receive partial/full credit.  Feel free to use Ed for clarification of any given question.  Always turn in \textit{only} your own, independent work.  Scan, upload, and \textbf{submit via Gradescope} (assign each question to a page in your submission).  For our late policy, refer to the syllabus.  All other administrative questions can be posted on Slack.
}

\vspace{0.5cm}

\textbf{Collaborators:}\\
%%%%%%%%%%%%%%%%%%%%%%%%%%%%%%%%%%%%%%%%%
%                                       %
% TODO: Names of any Collaborators Here %
%                                       %
%%%%%%%%%%%%%%%%%%%%%%%%%%%%%%%%%%%%%%%%%
\\
\textbf{AI Tool Disclosure:}
%%%%%%%%%%%%%%%%%%%%%%%%%%%%%%%%%%%%%%%%%
%                                       %
% TODO: How did you use AI tools?       %
%                                       %
%%%%%%%%%%%%%%%%%%%%%%%%%%%%%%%%%%%%%%%%%
\newpage

The following exercises will review some math and (re)introduce the type-setting tool \LaTeX. In addition to these written problems, you will also complete the coding part of the assignment, found in the GitHub repo.\\

\section{Algebra, the linear kind. (2 points)}
Solve the following system of linear equations for $x, y, \text{ and } z$. Show your work and use matrix algebra.
\colorbox[blue!5]{
\vspace{-3mm}
\begin{equation*}
    \begin{split}
        x + 2y + 3z &= 1 \\
        2x + 2y + 3z &= 2 \\
        x + 3y + 4z &= 2
    \end{split}.
\end{equation*}
}

Hint: the inverse of
    $\begin{bmatrix}
        1 & 2 & 3\\
        2 & 2 & 3\\
        1 & 3 & 4
        \end{bmatrix}$
    is
    $\begin{bmatrix}
        -1 & 1 & 0\\
        -5 & 1 & 3\\
        4 & -1 & -2
        \end{bmatrix}$

%%%%%%%%%%%%%%%%%%%%%%%%%%%%%%%%%%%%%%%%%
% TODO: Your solution to Problem 1      %
%                                       %
%                                       %
%                                       %
%                                       %
%%%%%%%%%%%%%%%%%%%%%%%%%%%%%%%%%%%%%%%%%
\newpage

\section{Calculus, the multivariable kind. (4 points)}
\subsection{Minimize! (2 points)}
Suppose that we wish to minimize some loss function $\mathcal{L}$ with respect to a variable $x$ where:
$$\mathcal{L}(x) = 3x^2 + 2x - 1$$

Given this loss function, what value of $x$ minimizes the loss?  In other words, for what value of $x$ is this function minimal?
%%%%%%%%%%%%%%%%%%%%%%%%%%%%%%%%%%%%%%%%%
% TODO: Your solution to Problem 2.1    %
%                                       %
%                                       %
%                                       %
%                                       %
%%%%%%%%%%%%%%%%%%%%%%%%%%%%%%%%%%%%%%%%%
\vspace{5cm}

\subsection{Differentiate! Partially.  (2 points)}
Find the partial derivative $\frac{\partial f}{\partial x}$ of the following function:
$$ f(x, y) = 3x^2 + 2xy - 5y^2 $$
%%%%%%%%%%%%%%%%%%%%%%%%%%%%%%%%%%%%%%%%%
% TODO: Your solution to Problem 2.2    %
%                        Part 1         %
%                                       %
%                                       %
%                                       %
%%%%%%%%%%%%%%%%%%%%%%%%%%%%%%%%%%%%%%%%%
\vspace{4cm}

Now evaluate that derivative at $(x,y) = (2,3)$.
%%%%%%%%%%%%%%%%%%%%%%%%%%%%%%%%%%%%%%%%%
% TODO: Your solution to Problem 2.2    %
%                        Part 2         %
%                                       %
%                                       %
%                                       %
%%%%%%%%%%%%%%%%%%%%%%%%%%%%%%%%%%%%%%%%%
\vspace{1.5cm}

\newpage
\section{Asymptotically! (2 points)}
In this class, we will loosely use Big-O asymptotic notation.  This notation provides a common vocabulary to discuss run-times of algorithms.
\begin{itemize}
\item For a simple beginner's guide, see \href{https://rob-bell.net/2009/06/a-beginners-guide-to-big-o-notation/}{here}.
\item Recall that the insertion sort algorithm takes $O(n^2)$ time---where $n$ is the length of the list to sort.
\item Merge sort takes $O(n\log n)$ time.
\end{itemize}

Consider the following code excerpts:
\begin{tcolorbox}[left=14pt, arc=0pt, outer arc=0pt, colframe=blue!5, colback=blue!5]
\begin{lstlisting}
def sum_list(lst):
    """
    Function that returns the sum of elements in lst
    :param lst: input list
    :type  lst: [int]
    :return: sum of the list
    :rtype : int
    """
    rtn = 0
    for elt in lst:
        rtn += elt
    return rtn


def foo(lst):
    """
    Foo Algorithm
    :param lst: input list of length n
    :type  lst: [int]
    :return: ???
    :rtype : int
    """
    for elt in lst:
        if elt < 0:
            return -1
    return sum_list(lst)
\end{lstlisting}
\end{tcolorbox}

\vspace{0.5cm}

What is the runtime of the \texttt{foo} algorithm in Big-O notation?
%%%%%%%%%%%%%%%%%%%%%%%%%%%%%%%%%%%%%%%%%
% TODO: Your solution to Problem 3      %
%                        Part 1         %
%                                       %
%                                       %
%                                       %
%%%%%%%%%%%%%%%%%%%%%%%%%%%%%%%%%%%%%%%%%
\vspace{2cm}


What does the \texttt{foo} function do?
%%%%%%%%%%%%%%%%%%%%%%%%%%%%%%%%%%%%%%%%%
% TODO: Your solution to Problem 3      %
%                        Part 2         %
%                                       %
%                                       %
%                                       %
%%%%%%%%%%%%%%%%%%%%%%%%%%%%%%%%%%%%%%%%%
\newpage


\section{The Old Man and the \texttt{C} (2 Points)}

Consider the following C code:

\begin{tcolorbox}[left=14pt, arc=0pt, outer arc=0pt, colframe=blue!5, colback=blue!5]
\begin{lstlisting}[language=c]
#include <stdio.h>

int main() {
    int arr[5] = {1, 2, 3, 4, 5};
    int* ptr = &arr[2];
    *ptr = 7;
    printf("%d\n", *ptr + 2);
    return 0;
}
\end{lstlisting}
\end{tcolorbox}

\vspace{0.5cm}

What values are in the final array? (\textbf{Note:} if you don't remember much about pointers and \texttt{C}/\texttt{C++}, try the coding part of PS0 first, then return to this problem and/or copy and run this code!)
%%%%%%%%%%%%%%%%%%%%%%%%%%%%%%%%%%%%%%%%%
% TODO: Your solution to Problem 4      %
%                        Part 1         %
%                                       %
%                                       %
%                                       %
%%%%%%%%%%%%%%%%%%%%%%%%%%%%%%%%%%%%%%%%%
\vspace{3cm}

What is printed out by the function?
%%%%%%%%%%%%%%%%%%%%%%%%%%%%%%%%%%%%%%%%%
% TODO: Your solution to Problem 4      %
%                        Part 2         %
%                                       %
%                                       %
%                                       %
%%%%%%%%%%%%%%%%%%%%%%%%%%%%%%%%%%%%%%%%%

\vfill
\begin{tiny}Updated \today \end{tiny}
\end{document}